\documentclass[10pt]{article}
\input{../aux/encabezado.tex}
\input{../aux/tikzlibrarybayesnet.code.tex}

\newif\ifen
\newif\ifes
\newcommand{\en}[1]{\ifen#1\fi}
\newcommand{\es}[1]{\ifes#1\fi}
\entrue

\title{\huge
\en{Exploiting the properties of the epistemic-evolutionary cost function}
\es{Explotabdo las propiedades de la función de costo epistémico-evolutiva}
}

\author{Gustavo Landfried$^{1,2}$}
\affil{\small 1. Bayes de las Provincias Unidas del Sur }
\affil{\vspace{-0.2cm}\small 2. Laboratorio Pacha Pampas}
\affil[]{Correspondencia: \texttt{glandfried@dc.uba.ar}, \texttt{bayesdelsur@gmail.com}}

\begin{document}

\maketitle

\begin{abstract}
\en{Both evolutionary and probabilistic selection processes are multiplicative in nature. }%
\es{Tanto los procesos de selección evolutiva como los probabilísticos son de naturaleza multiplicativa. }%
%
\en{In evolutionary theory, everyone is taught that lineage growth follows a noisy, multiplicative process: a sequence of survival and reproduction rates. }%
\es{En la teoría evolutiva todo el mundo aprende que el crecimiento de los linaje sigue un proceso multiplicativo y ruidoso: una productoria de tasas de supervivencia y reproducción. }%
%
\en{In probability theory, all alternative axiomatisations conclude that hypotheses must be selected by the product rule: the multiplication of prior predictions of observed data. }%
\es{En la teoría de la probabilidad, todas las axiomatizaciones alternativas llegan a la conclusión de que las hipótesis se seleccionan mediante la regla del producto: la productoria de predicciones a priori de los datos observados. }%
%
\en{Since in multiplicative processes the impacts of losses are stronger than gains there is an advantage in favour of variants that reduce fluctuations: individual diversification (epistemic property), cooperation (major evolutionary property), specialisation (meta-epistemic property), coexistence (ecological property)~\cite{landfried2022-transitions}. }%
\es{Dado que en los procesos multiplicativos los impactos de las pérdidas son más fuertes que los de las ganancias existe una ventaja a favor de las variantes que reducen las fluctuaciones: diversificación individual (propiedad epistémica), cooperación (propiedad evolutiva mayor), especialización (propiedad meta-epistémica), coexistencia (propiedad ecológica)~\cite{landfried2022-transitions}. }%
%
\en{Individually the growth rate is maximised by dividing the resources in the same proportion as the observed frequency. }%
\es{Individualmente la tasa de crecimiento se maximiza dividiendo los recursos en la misma proporción que la frecuencia observada. }%
\en{This property is used by probability theory to acquire knowledge through observation. }%
\es{Esta propiedad es utilizada por la teoría de la probabilidad para adquirir conocimineto a partir de la observación. }%
%
\en{However, cooperation causes an increase in the growth rate, which is not possible to achieve individually. }%
\es{Sin embargo, la cooperación produce un aumento de la tasa de crecimiento, que no es posible conseguir individualmente. }%
%
\en{This property drives the emergence of major evolutionary transitions. }%
\es{Esta propiedad impulsa la emergencia de las transiciones evolutivas mayores. }%
%
\en{The cultural transition produced radical changes for our species, allowing us to occupy all the ecological niches of the earth as no other terrestrial vertebrate had ever done before. }%
\es{La transición cultural produjo cambios radicales para nuestra especie, ella nos permitió ocupar todos los nichos ecológicos de la tierra como ningún otro vertebrado terrestre lo había logrado antes. }%
%
\en{Because culture is a common resource, an advantage in favor of specialist variants arises: science develops faster when we invest all our resources in the great discoveries, even if most of us fail. }%
\es{Gracias a que la cultura es un recurso común, aparece una ventaja a favor de las variantes especialistas: la ciencia se desarrolla más rápido cuando invertimos todos nuestros recursos en los grandes descubrimientos a pesar de que la mayoría fracasemos. }%
%
\en{However, empirical analyses of cultural evolution have concluded that a degree of cultural diversity needs to be preserved in order not to produce a negative impact on innovation, which would be achieved by a 'partially connected' network structure~\cite{derex2020-evolvingStructure}. }%
\es{Sin embargo, análisis empíricos de evolución cultural han llegado a la conclusión de que es necesario preservar un grado de diversidad cultural para no producir un impacto negativo en la innovación, lo que se obtendría con una estructura de red ``parcialmente conectada''~\cite{derex2020-evolvingStructure}. }%
%
\en{Indeed, human societies typically develop partially connected networks that allow them to acquire relatively large effective population sizes while maintaining local differentiation~\cite{migliano2017-hunterGathererNetwork, padilla2022-interconnectivity}. }%
\es{En efecto, las sociedades humanas desarrollan típicamente redes parcialmente conectadas que permite adquirir tamaños de población efectiva relativamente grandes menteniendo al mismo tiempo una diferenciación local~\cite{migliano2017-hunterGathererNetwork, padilla2022-interconnectivity}. }%
%
\en{Thus, through the coexistence of diverse cooperative-specialist units, humans would have exploited the four properties of evolutionary selection processes. }%
\es{De esta forma, a través de la coexistencia de diversas unidades cooperativas-especialistas, los humanos habríamos explotado las cuatro propiedades de los procesos de selección evolutiva. }%

\vspace{0.1cm}

Lightning talk + Poster presentation

\end{abstract}


{\footnotesize \bibliographystyle{../aux/biblio/plos2015}
\bibliography{../aux/biblio/biblio_notUrl}
}

\end{document}
