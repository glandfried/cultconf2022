\documentclass[10pt]{article}
\usepackage{fullpage}
\usepackage{cite}
\usepackage[utf8]{inputenc}
\usepackage{a4wide}
\usepackage{url}
\usepackage{graphicx}
\usepackage{caption}
\usepackage{float} % para que los gr\'aficos se queden en su lugar con [H]
\usepackage{subcaption}
\usepackage{wrapfig}
\usepackage{color}
\usepackage{amsmath} %para escribir funci\'on partida , matrices
\usepackage{amsthm} %para numerar definciones y teoremas
\usepackage[hidelinks]{hyperref} % para inlcuir links dentro del texto
\usepackage{tabu} 
\usepackage{comment}
\usepackage{amsfonts} % \mathbb{N} -> conjunto de los n\'umeros naturales  
\usepackage{enumerate}
\usepackage{listings}
\usepackage[colorinlistoftodos, textsize=small]{todonotes} % Para poner notas en el medio del texto!! No olvidar hacer. 
\usepackage{framed} % Para encuadrar texto. \begin{framed}
\usepackage{csquotes} % Para citar texto \begin{displayquote}
\usepackage{epigraph} % Epigrafe  \epigraph{texto}{\textit{autor}}
\usepackage{authblk}
\usepackage{titlesec}
\usepackage{varioref}
\usepackage{bm} % \bm{\alpha} bold greek symbol
\usepackage{pdfpages} % \includepdf
\usepackage[makeroom]{cancel} % \cancel{} \bcancel{} etc
\usepackage{wrapfig} % \begin{wrapfigure} Pone figura al lado del texto
\usepackage{mdframed}
\usepackage{algorithm}
%\usepackage{quoting}
\usepackage{mathtools}	
\usepackage{tikz}
\usepackage{paracol}

\newcommand{\vm}[1]{\mathbf{#1}}
\newcommand{\N}{\mathcal{N}}
\newcommand{\citel}[1]{\cite{#1}\label{#1}}
\newcommand\hfrac[2]{\genfrac{}{}{0pt}{}{#1}{#2}} %\frac{}{} sin la linea del medio

\newtheorem{midef}{Definition}
\newtheorem{miteo}{Theorem}
\newtheorem{mipropo}{Proposition}

\theoremstyle{definition}
\newtheorem{definition}{Definition}[section]
\newtheorem{theorem}{Theorem}[section]
\newtheorem{proposition}{Proposition}[section]


%http://latexcolor.com/
\definecolor{azul}{rgb}{0.36, 0.54, 0.66}
\definecolor{rojo}{rgb}{0.7, 0.2, 0.116}
\definecolor{rojopiso}{rgb}{0.8, 0.25, 0.17}
\definecolor{verdeingles}{rgb}{0.12, 0.5, 0.17}
\definecolor{ubuntu}{rgb}{0.44, 0.16, 0.39}
\definecolor{debian}{rgb}{0.84, 0.04, 0.33}
\definecolor{dkgreen}{rgb}{0,0.6,0}
\definecolor{gray}{rgb}{0.5,0.5,0.5}
\definecolor{mauve}{rgb}{0.58,0,0.82}

\lstset{
  language=Python,
  aboveskip=3mm,
  belowskip=3mm,
  showstringspaces=true,
  columns=flexible,
  basicstyle={\small\ttfamily},
  numbers=none,
  numberstyle=\tiny\color{gray},
  keywordstyle=\color{blue},
  commentstyle=\color{dkgreen},
  stringstyle=\color{mauve},
  breaklines=true,
  breakatwhitespace=true,
  tabsize=4
}

% tikzlibrary.code.tex
%
% Copyright 2010-2011 by Laura Dietz
% Copyright 2012 by Jaakko Luttinen
%
% This file may be distributed and/or modified
%
% 1. under the LaTeX Project Public License and/or
% 2. under the GNU General Public License.
%
% See the files LICENSE_LPPL and LICENSE_GPL for more details.

% Load other libraries

%\newcommand{\vast}{\bBigg@{2.5}}
% newcommand{\Vast}{\bBigg@{14.5}}
% \usepackage{helvet}
% \renewcommand{\familydefault}{\sfdefault}

\usetikzlibrary{shapes}
\usetikzlibrary{fit}
\usetikzlibrary{chains}
\usetikzlibrary{arrows}

% Latent node
\tikzstyle{latent} = [circle,fill=white,draw=black,inner sep=1pt,
minimum size=20pt, font=\fontsize{10}{10}\selectfont, node distance=1]
% Observed node
\tikzstyle{obs} = [latent,fill=gray!25]
% Invisible node
\tikzstyle{invisible} = [latent,minimum size=0pt,color=white, opacity=0, node distance=0]
% Constant node
\tikzstyle{const} = [rectangle, inner sep=0pt, node distance=0.1]
%state
\tikzstyle{estado} = [latent,minimum size=8pt,node distance=0.4]
%action
\tikzstyle{accion} =[latent,circle,minimum size=5pt,fill=black,node distance=0.4]


% Factor node
\tikzstyle{factor} = [rectangle, fill=black,minimum size=10pt, draw=black, inner
sep=0pt, node distance=1]
% Deterministic node
\tikzstyle{det} = [latent, rectangle]

% Plate node
\tikzstyle{plate} = [draw, rectangle, rounded corners, fit=#1]
% Invisible wrapper node
\tikzstyle{wrap} = [inner sep=0pt, fit=#1]
% Gate
\tikzstyle{gate} = [draw, rectangle, dashed, fit=#1]

% Caption node
\tikzstyle{caption} = [font=\footnotesize, node distance=0] %
\tikzstyle{plate caption} = [caption, node distance=0, inner sep=0pt,
below left=5pt and 0pt of #1.south east] %
\tikzstyle{factor caption} = [caption] %
\tikzstyle{every label} += [caption] %

\tikzset{>={triangle 45}}

%\pgfdeclarelayer{b}
%\pgfdeclarelayer{f}
%\pgfsetlayers{b,main,f}

% \factoredge [options] {inputs} {factors} {outputs}
\newcommand{\factoredge}[4][]{ %
  % Connect all nodes #2 to all nodes #4 via all factors #3.
  \foreach \f in {#3} { %
    \foreach \x in {#2} { %
      \path (\x) edge[-,#1] (\f) ; %
      %\draw[-,#1] (\x) edge[-] (\f) ; %
    } ;
    \foreach \y in {#4} { %
      \path (\f) edge[->,#1] (\y) ; %
      %\draw[->,#1] (\f) -- (\y) ; %
    } ;
  } ;
}

% \edge [options] {inputs} {outputs}
\newcommand{\edge}[3][]{ %
  % Connect all nodes #2 to all nodes #3.
  \foreach \x in {#2} { %
    \foreach \y in {#3} { %
      \path (\x) edge [->,#1] (\y) ;%
      %\draw[->,#1] (\x) -- (\y) ;%
    } ;
  } ;
}

% \factor [options] {name} {caption} {inputs} {outputs}
\newcommand{\factor}[5][]{ %
  % Draw the factor node. Use alias to allow empty names.
  \node[factor, label={[name=#2-caption]#3}, name=#2, #1,
  alias=#2-alias] {} ; %
  % Connect all inputs to outputs via this factor
  \factoredge {#4} {#2-alias} {#5} ; %
}

% \plate [options] {name} {fitlist} {caption}
\newcommand{\plate}[4][]{ %
  \node[wrap=#3] (#2-wrap) {}; %
  \node[plate caption=#2-wrap] (#2-caption) {#4}; %
  \node[plate=(#2-wrap)(#2-caption), #1] (#2) {}; %
}

% \gate [options] {name} {fitlist} {inputs}
\newcommand{\gate}[4][]{ %
  \node[gate=#3, name=#2, #1, alias=#2-alias] {}; %
  \foreach \x in {#4} { %
    \draw [-*,thick] (\x) -- (#2-alias); %
  } ;%
}

% \vgate {name} {fitlist-left} {caption-left} {fitlist-right}
% {caption-right} {inputs}
\newcommand{\vgate}[6]{ %
  % Wrap the left and right parts
  \node[wrap=#2] (#1-left) {}; %
  \node[wrap=#4] (#1-right) {}; %
  % Draw the gate
  \node[gate=(#1-left)(#1-right)] (#1) {}; %
  % Add captions
  \node[caption, below left=of #1.north ] (#1-left-caption)
  {#3}; %
  \node[caption, below right=of #1.north ] (#1-right-caption)
  {#5}; %
  % Draw middle separation
  \draw [-, dashed] (#1.north) -- (#1.south); %
  % Draw inputs
  \foreach \x in {#6} { %
    \draw [-*,thick] (\x) -- (#1); %
  } ;%
}

% \hgate {name} {fitlist-top} {caption-top} {fitlist-bottom}
% {caption-bottom} {inputs}
\newcommand{\hgate}[6]{ %
  % Wrap the left and right parts
  \node[wrap=#2] (#1-top) {}; %
  \node[wrap=#4] (#1-bottom) {}; %
  % Draw the gate
  \node[gate=(#1-top)(#1-bottom)] (#1) {}; %
  % Add captions
  \node[caption, above right=of #1.west ] (#1-top-caption)
  {#3}; %
  \node[caption, below right=of #1.west ] (#1-bottom-caption)
  {#5}; %
  % Draw middle separation
  \draw [-, dashed] (#1.west) -- (#1.east); %
  % Draw inputs
  \foreach \x in {#6} { %
    \draw [-*,thick] (\x) -- (#1); %
  } ;%
}



\newif\ifen
\newif\ifes
\newcommand{\en}[1]{\ifen#1\fi}
\newcommand{\es}[1]{\ifes#1\fi}
\entrue

\title{\huge
\en{Exploiting the properties of the epistemic-evolutionary cost function}
\es{Explotabdo las propiedades de la función de costo epistémico-evolutiva}
}

\author{Gustavo Landfried$^{1,2}$}
\affil{\small 1. Bayes de las Provincias Unidas del Sur }
\affil{\vspace{-0.2cm}\small 2. Laboratorio Pacha Pampas}
\affil[]{Correspondencia: \texttt{glandfried@dc.uba.ar}, \texttt{bayesdelsur@gmail.com}}

\begin{document}

\maketitle

\begin{abstract}
\en{Both evolutionary and probabilistic selection processes are multiplicative in nature. }%
\es{Tanto los procesos de selección evolutiva como los probabilísticos son de naturaleza multiplicativa. }%
%
\en{In evolutionary theory, everyone is taught that lineage growth follows a noisy, multiplicative process: a sequence of survival and reproduction rates. }%
\es{En la teoría evolutiva todo el mundo aprende que el crecimiento de los linaje sigue un proceso multiplicativo y ruidoso: una productoria de tasas de supervivencia y reproducción. }%
%
\en{In probability theory, all alternative axiomatisations conclude that hypotheses must be selected by the product rule: the multiplication of prior predictions of observed data. }%
\es{En la teoría de la probabilidad, todas las axiomatizaciones alternativas llegan a la conclusión de que las hipótesis se seleccionan mediante la regla del producto: la productoria de predicciones a priori de los datos observados. }%
%
\en{Since in multiplicative processes the impacts of losses are stronger than gains there is an advantage in favour of variants that reduce fluctuations: individual diversification (epistemic property), cooperation (major evolutionary property), specialisation (meta-epistemic property), coexistence (ecological property)~\cite{landfried2022-transitions}. }%
\es{Dado que en los procesos multiplicativos los impactos de las pérdidas son más fuertes que los de las ganancias existe una ventaja a favor de las variantes que reducen las fluctuaciones: diversificación individual (propiedad epistémica), cooperación (propiedad evolutiva mayor), especialización (propiedad meta-epistémica), coexistencia (propiedad ecológica)~\cite{landfried2022-transitions}. }%
%
\en{Individually the growth rate is maximised by dividing the resources in the same proportion as the observed frequency. }%
\es{Individualmente la tasa de crecimiento se maximiza dividiendo los recursos en la misma proporción que la frecuencia observada. }%
\en{This property is used by probability theory to acquire knowledge through observation. }%
\es{Esta propiedad es utilizada por la teoría de la probabilidad para adquirir conocimineto a partir de la observación. }%
%
\en{However, cooperation causes an increase in the growth rate, which is not possible to achieve individually. }%
\es{Sin embargo, la cooperación produce un aumento de la tasa de crecimiento, que no es posible conseguir individualmente. }%
%
\en{This property drives the emergence of major evolutionary transitions. }%
\es{Esta propiedad impulsa la emergencia de las transiciones evolutivas mayores. }%
%
\en{The cultural transition produced radical changes for our species, allowing us to occupy all the ecological niches of the earth as no other terrestrial vertebrate had ever done before. }%
\es{La transición cultural produjo cambios radicales para nuestra especie, ella nos permitió ocupar todos los nichos ecológicos de la tierra como ningún otro vertebrado terrestre lo había logrado antes. }%
%
\en{Because culture is a common resource, an advantage in favor of specialist variants arises: science develops faster when we invest all our resources in the great discoveries, even if most of us fail. }%
\es{Gracias a que la cultura es un recurso común, aparece una ventaja a favor de las variantes especialistas: la ciencia se desarrolla más rápido cuando invertimos todos nuestros recursos en los grandes descubrimientos a pesar de que la mayoría fracasemos. }%
%
\en{However, empirical analyses of cultural evolution have concluded that a degree of cultural diversity needs to be preserved in order not to produce a negative impact on innovation, which would be achieved by a 'partially connected' network structure~\cite{derex2020-evolvingStructure}. }%
\es{Sin embargo, análisis empíricos de evolución cultural han llegado a la conclusión de que es necesario preservar un grado de diversidad cultural para no producir un impacto negativo en la innovación, lo que se obtendría con una estructura de red ``parcialmente conectada''~\cite{derex2020-evolvingStructure}. }%
%
\en{Indeed, human societies typically develop partially connected networks that allow them to acquire relatively large effective population sizes while maintaining local differentiation~\cite{migliano2017-hunterGathererNetwork, padilla2022-interconnectivity}. }%
\es{En efecto, las sociedades humanas desarrollan típicamente redes parcialmente conectadas que permite adquirir tamaños de población efectiva relativamente grandes menteniendo al mismo tiempo una diferenciación local~\cite{migliano2017-hunterGathererNetwork, padilla2022-interconnectivity}. }%
%
\en{Thus, through the coexistence of diverse cooperative-specialist units, humans would have exploited the four properties of evolutionary selection processes. }%
\es{De esta forma, a través de la coexistencia de diversas unidades cooperativas-especialistas, los humanos habríamos explotado las cuatro propiedades de los procesos de selección evolutiva. }%

\vspace{0.1cm}

Lightning talk + Poster presentation

\end{abstract}


{\footnotesize \bibliographystyle{../aux/biblio/plos2015}
\bibliography{../aux/biblio/biblio_notUrl}
}

\end{document}
