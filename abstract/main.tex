\documentclass[10pt]{article}
\usepackage{fullpage}
\usepackage{cite}
\usepackage[utf8]{inputenc}
\usepackage{a4wide}
\usepackage{url}
\usepackage{graphicx}
\usepackage{caption}
\usepackage{float} % para que los gr\'aficos se queden en su lugar con [H]
\usepackage{subcaption}
\usepackage{wrapfig}
\usepackage{color}
\usepackage{amsmath} %para escribir funci\'on partida , matrices
\usepackage{amsthm} %para numerar definciones y teoremas
\usepackage[hidelinks]{hyperref} % para inlcuir links dentro del texto
\usepackage{tabu} 
\usepackage{comment}
\usepackage{amsfonts} % \mathbb{N} -> conjunto de los n\'umeros naturales  
\usepackage{enumerate}
\usepackage{listings}
\usepackage[colorinlistoftodos, textsize=small]{todonotes} % Para poner notas en el medio del texto!! No olvidar hacer. 
\usepackage{framed} % Para encuadrar texto. \begin{framed}
\usepackage{csquotes} % Para citar texto \begin{displayquote}
\usepackage{epigraph} % Epigrafe  \epigraph{texto}{\textit{autor}}
\usepackage{authblk}
\usepackage{titlesec}
\usepackage{varioref}
\usepackage{bm} % \bm{\alpha} bold greek symbol
\usepackage{pdfpages} % \includepdf
\usepackage[makeroom]{cancel} % \cancel{} \bcancel{} etc
\usepackage{wrapfig} % \begin{wrapfigure} Pone figura al lado del texto
\usepackage{mdframed}
\usepackage{algorithm}
%\usepackage{quoting}
\usepackage{mathtools}	
\usepackage{tikz}
\usepackage{paracol}

\newcommand{\vm}[1]{\mathbf{#1}}
\newcommand{\N}{\mathcal{N}}
\newcommand{\citel}[1]{\cite{#1}\label{#1}}
\newcommand\hfrac[2]{\genfrac{}{}{0pt}{}{#1}{#2}} %\frac{}{} sin la linea del medio

\newtheorem{midef}{Definition}
\newtheorem{miteo}{Theorem}
\newtheorem{mipropo}{Proposition}

\theoremstyle{definition}
\newtheorem{definition}{Definition}[section]
\newtheorem{theorem}{Theorem}[section]
\newtheorem{proposition}{Proposition}[section]


%http://latexcolor.com/
\definecolor{azul}{rgb}{0.36, 0.54, 0.66}
\definecolor{rojo}{rgb}{0.7, 0.2, 0.116}
\definecolor{rojopiso}{rgb}{0.8, 0.25, 0.17}
\definecolor{verdeingles}{rgb}{0.12, 0.5, 0.17}
\definecolor{ubuntu}{rgb}{0.44, 0.16, 0.39}
\definecolor{debian}{rgb}{0.84, 0.04, 0.33}
\definecolor{dkgreen}{rgb}{0,0.6,0}
\definecolor{gray}{rgb}{0.5,0.5,0.5}
\definecolor{mauve}{rgb}{0.58,0,0.82}

\lstset{
  language=Python,
  aboveskip=3mm,
  belowskip=3mm,
  showstringspaces=true,
  columns=flexible,
  basicstyle={\small\ttfamily},
  numbers=none,
  numberstyle=\tiny\color{gray},
  keywordstyle=\color{blue},
  commentstyle=\color{dkgreen},
  stringstyle=\color{mauve},
  breaklines=true,
  breakatwhitespace=true,
  tabsize=4
}

% tikzlibrary.code.tex
%
% Copyright 2010-2011 by Laura Dietz
% Copyright 2012 by Jaakko Luttinen
%
% This file may be distributed and/or modified
%
% 1. under the LaTeX Project Public License and/or
% 2. under the GNU General Public License.
%
% See the files LICENSE_LPPL and LICENSE_GPL for more details.

% Load other libraries

%\newcommand{\vast}{\bBigg@{2.5}}
% newcommand{\Vast}{\bBigg@{14.5}}
% \usepackage{helvet}
% \renewcommand{\familydefault}{\sfdefault}

\usetikzlibrary{shapes}
\usetikzlibrary{fit}
\usetikzlibrary{chains}
\usetikzlibrary{arrows}

% Latent node
\tikzstyle{latent} = [circle,fill=white,draw=black,inner sep=1pt,
minimum size=20pt, font=\fontsize{10}{10}\selectfont, node distance=1]
% Observed node
\tikzstyle{obs} = [latent,fill=gray!25]
% Invisible node
\tikzstyle{invisible} = [latent,minimum size=0pt,color=white, opacity=0, node distance=0]
% Constant node
\tikzstyle{const} = [rectangle, inner sep=0pt, node distance=0.1]
%state
\tikzstyle{estado} = [latent,minimum size=8pt,node distance=0.4]
%action
\tikzstyle{accion} =[latent,circle,minimum size=5pt,fill=black,node distance=0.4]


% Factor node
\tikzstyle{factor} = [rectangle, fill=black,minimum size=10pt, draw=black, inner
sep=0pt, node distance=1]
% Deterministic node
\tikzstyle{det} = [latent, rectangle]

% Plate node
\tikzstyle{plate} = [draw, rectangle, rounded corners, fit=#1]
% Invisible wrapper node
\tikzstyle{wrap} = [inner sep=0pt, fit=#1]
% Gate
\tikzstyle{gate} = [draw, rectangle, dashed, fit=#1]

% Caption node
\tikzstyle{caption} = [font=\footnotesize, node distance=0] %
\tikzstyle{plate caption} = [caption, node distance=0, inner sep=0pt,
below left=5pt and 0pt of #1.south east] %
\tikzstyle{factor caption} = [caption] %
\tikzstyle{every label} += [caption] %

\tikzset{>={triangle 45}}

%\pgfdeclarelayer{b}
%\pgfdeclarelayer{f}
%\pgfsetlayers{b,main,f}

% \factoredge [options] {inputs} {factors} {outputs}
\newcommand{\factoredge}[4][]{ %
  % Connect all nodes #2 to all nodes #4 via all factors #3.
  \foreach \f in {#3} { %
    \foreach \x in {#2} { %
      \path (\x) edge[-,#1] (\f) ; %
      %\draw[-,#1] (\x) edge[-] (\f) ; %
    } ;
    \foreach \y in {#4} { %
      \path (\f) edge[->,#1] (\y) ; %
      %\draw[->,#1] (\f) -- (\y) ; %
    } ;
  } ;
}

% \edge [options] {inputs} {outputs}
\newcommand{\edge}[3][]{ %
  % Connect all nodes #2 to all nodes #3.
  \foreach \x in {#2} { %
    \foreach \y in {#3} { %
      \path (\x) edge [->,#1] (\y) ;%
      %\draw[->,#1] (\x) -- (\y) ;%
    } ;
  } ;
}

% \factor [options] {name} {caption} {inputs} {outputs}
\newcommand{\factor}[5][]{ %
  % Draw the factor node. Use alias to allow empty names.
  \node[factor, label={[name=#2-caption]#3}, name=#2, #1,
  alias=#2-alias] {} ; %
  % Connect all inputs to outputs via this factor
  \factoredge {#4} {#2-alias} {#5} ; %
}

% \plate [options] {name} {fitlist} {caption}
\newcommand{\plate}[4][]{ %
  \node[wrap=#3] (#2-wrap) {}; %
  \node[plate caption=#2-wrap] (#2-caption) {#4}; %
  \node[plate=(#2-wrap)(#2-caption), #1] (#2) {}; %
}

% \gate [options] {name} {fitlist} {inputs}
\newcommand{\gate}[4][]{ %
  \node[gate=#3, name=#2, #1, alias=#2-alias] {}; %
  \foreach \x in {#4} { %
    \draw [-*,thick] (\x) -- (#2-alias); %
  } ;%
}

% \vgate {name} {fitlist-left} {caption-left} {fitlist-right}
% {caption-right} {inputs}
\newcommand{\vgate}[6]{ %
  % Wrap the left and right parts
  \node[wrap=#2] (#1-left) {}; %
  \node[wrap=#4] (#1-right) {}; %
  % Draw the gate
  \node[gate=(#1-left)(#1-right)] (#1) {}; %
  % Add captions
  \node[caption, below left=of #1.north ] (#1-left-caption)
  {#3}; %
  \node[caption, below right=of #1.north ] (#1-right-caption)
  {#5}; %
  % Draw middle separation
  \draw [-, dashed] (#1.north) -- (#1.south); %
  % Draw inputs
  \foreach \x in {#6} { %
    \draw [-*,thick] (\x) -- (#1); %
  } ;%
}

% \hgate {name} {fitlist-top} {caption-top} {fitlist-bottom}
% {caption-bottom} {inputs}
\newcommand{\hgate}[6]{ %
  % Wrap the left and right parts
  \node[wrap=#2] (#1-top) {}; %
  \node[wrap=#4] (#1-bottom) {}; %
  % Draw the gate
  \node[gate=(#1-top)(#1-bottom)] (#1) {}; %
  % Add captions
  \node[caption, above right=of #1.west ] (#1-top-caption)
  {#3}; %
  \node[caption, below right=of #1.west ] (#1-bottom-caption)
  {#5}; %
  % Draw middle separation
  \draw [-, dashed] (#1.west) -- (#1.east); %
  % Draw inputs
  \foreach \x in {#6} { %
    \draw [-*,thick] (\x) -- (#1); %
  } ;%
}



\newif\ifen
\newif\ifes
\newcommand{\en}[1]{\ifen#1\fi}
\newcommand{\es}[1]{\ifes#1\fi}
\entrue

\title{\huge
\en{Properties of the epistemic-evolutionary cost function}
\es{Propiedades de la función de costo epistémico-evolutiva}
}

\author{Gustavo Landfried$^{1,2}$}
\affil{\small 1. Bayes de las Provincias Unidas del Sur }
\affil{\vspace{-0.2cm}\small 2. Laboratorio Pacha Pampas}
\affil[]{Correspondencia: \texttt{glandfried@dc.uba.ar}, \texttt{bayesdelsur@gmail.com}}

\begin{document}

\maketitle

\section{Abstract}

\en{Both evolutionary and probabilistic selection processes are multiplicative in nature. }%
\es{Tanto los procesos de selección evolutiva como los probabilísticos son de naturaleza multiplicativa. }%
%
\en{In evolutionary theory, everyone is taught that lineage growth follows a noisy, multiplicative process: a sequence of survival and reproduction rates. }%
\es{En la teoría evolutiva todo el mundo aprende que el crecimiento de los linaje sigue un proceso multiplicativo y ruidoso: una productoria de tasas de supervivencia y reproducción. }%
%
\es{In probability theory, all alternative axiomatisations conclude that hypotheses must be selected by the product rule: the multiplication of prior predictions of observed data. }%
\en{En la teoría de la probabilidad, todas las axiomatizaciones alternativas llegan a la conclusión de que las hipótesis se seleccionan mediante la regla del producto: la productoria de predicciones a priori de los datos observados. }%
%
\en{Since in multiplicative processes the impacts of losses are stronger than gains (e.g. a single zero is enough to generate an extinction) there is an advantage in favour of variants that reduce fluctuations: individual diversification (epistemic property), cooperation (major evolutionary property), specialisation (meta-epistemic property), coexistence (ecological property). }%
\es{Dado que en los procesos multiplicativos los impactos de las pérdidas son más fuertes que los de las ganancias (e.g. un solo cero es suficiente para generar una extinción) existe una ventaja a favor de las variantes que reducen las fluctuaciones: diversificación individual (propiedad epistémica), cooperación (propiedad evolutiva mayor), especialización (propiedad meta-epistémica), coexistencia (propiedad ecológica). }%
%
\en{Individually the growth rate is maximised by dividing the resources in the same proportion as the observed frequency. }%
\es{Individualmente la tasa de crecimiento se maximiza dividiendo los recursos en la misma proporción que la frecuencia observada. }%
\en{This is the property that probability theory uses to acquire knowledge through observation. }%
\es{Esta es la propiedad que utiliza la teoría de la probabilidad para adquirir conocimineto del mundo a partir de la observación. }%
%
\en{However, cooperation causes an increase in the growth rate, which is not possible to achieve individually. }%
\es{Sin embargo, la cooperación produce un aumento de la tasa de crecimiento, que no es posible conseguir individualmente. }%
%
\en{This is the property that drives the emergence of major evolutionary transitions. }%
\es{Esta es la propiedad que impulsa la emergencia de las transiciones evolutivas mayores. }%
%
\en{The emergence of culture produced radical changes for our species: before the cultural transition, we were in serious danger of extinction; after the cultural transition, our species occupied every ecological niche on earth as no other terrestrial vertebrate had ever done before. }%
\es{El surgimiento de la cultura produjo cambios radicales para nuestra especie: antes de la transición cultural, estuvimos en grave peligro de extinción; luego de la transición cultural, nuestra especie ocupó todos los nichos ecológicos de la tierra como ningún otro vertebrado terrestre lo había logrado antes. }%
%
%
\en{Since it is no longer necessary to reduce fluctuations individually through cooperation, an additional advantage for the specialist variants arises. }%
\es{Debido a que con la cooperación deja de ser necesario reducir las fluctuaciones individualmente, aparece una ventaja adicional a favor de las variantes especialistas. }%
%
%Esta propiedad es la que nos obliga como científicos a dedicar todos nuestros esfuerzos en unos pocos temas de investigación.
%
\en{Because culture is a common resource, the growth rate is maximised when we invest all our resources in great discoveries even though most of them will fail. }%
\es{Gracias a que la cultura es un recurso común, la tasa de crecimiento se maximiza cuando apostamos todos nuestros recursos en los grandes descubrimientos a pesar de que la mayoría fracasemos en el intento. }%
%
\en{However, specialisation can have a negative impact on innovation due to reduced cultural diversity. }%
\es{Sin embargo, la especialización puede tener un impacto negativo en la innovación debido a la reducción de la diversidad cultural. }%
%
\en{In this sense, it seems important to preserve diversity through the existence of partially independent cooperative units. }%
\es{En ese sentido parece importante preservar la diversidad a través de existencia de unidades cooperativas parcialmente independiente. }%
%
\en{This is in line with the conclusion reached by empirical analyses of networks and cultural evolution: the 'partially connected' structure is the one that maximises the rate of cultural growth. }%
\es{Esto coincide con la conclusión a la que han llegado los análisis empíricos de redes y evolución cultural: la estructura ``parcialmente conectada'' es la que maximiza la tasa de crecimiento cultural. }%
%
\en{As a species we need the coexistence of diverse cooperative-specialist units, exploiting the four properties of evolutionary selection processes. }%
\es{Como especie necesitamos la coexistencia de diversas unidades cooperativas-especialistas, explotando las cuatro propiedades de los procesos de selección evolutiva. }%







% En la historia de la vida también se observa una ventaja a favor de la coexistencia. Para una especie heterótrofa como la nuestra, la coexistencia con otras especies es la única estrategia posible. Las tecnologías de reciprocidad ecológica produjeron la aparición independiente de la domesticación de especies animales y vegetales. Todos los pueblos del mundo desarrollaron obligaciones de reciprocidad, de dar y de recibir. En particular, la teoría de la probabilidad nace en 1654 como una tecnología de reciprocidad: existe un precio justo que garantiza la coexistencia en contextos de incertidumbre.
% La ruptura de la coexistencia intercultural generó procesos de involución cultural, como en Tasmania durante el Holoceno y en Europa occidental durante la Edad Media. Si bien la interacción intercultural de la modernidad aceleró las innovaciones, el proceso colonial asociado produjo una ruptura radical especialmente a partir de la Era de genocidios que se inicia a mediados del siglo 19. A pesar de todos los avances, hoy la ciencia metropolitana no es capaz de revertir la masiva pérdida biodiversidad producto de la masiva pérdida de diversidad cultural previa.
% El objetivo del laboratorio Pacha Pampas son las tecnologías de reciprocidad y coexistencia.
% 




{\bibliographystyle{../aux/biblio/plos2015}
\bibliography{../aux/biblio/biblio_notUrl.bib}
}

\end{document}
