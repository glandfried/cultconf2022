Both evolutionary and probabilistic selection processes are multiplicative in nature.
In evolutionary theory, everyone is taught that lineage growth follows a noisy, multiplicative process: a sequence of survival and reproduction rates.
In probability theory, all alternative axiomatisations conclude that hypotheses must be selected by the product rule: the multiplication of prior predictions of observed data.
Since in multiplicative processes the impacts of losses are stronger than gains there is an advantage in favour of variants that reduce fluctuations: individual diversification (epistemic property), cooperation (major evolutionary property), specialisation (meta-epistemic property), coexistence (ecological property).
Individually the growth rate is maximised by dividing the resources in the same proportion as the observed frequency.
This property is used by probability theory to acquire knowledge through observation.
However, cooperation causes an increase in the growth rate, which is not possible to achieve individually.
This property drives the emergence of major evolutionary transitions.
The cultural transition produced radical changes for our species, allowing us to occupy all the ecological niches of the earth as no other terrestrial vertebrate had ever done before.
Because culture is a common resource, an advantage in favor of specialist variants arises: science develops faster when we invest all our resources in the great discoveries, even if most of us fail.
However, empirical analyses of cultural evolution have concluded that a degree of cultural diversity needs to be preserved in order not to produce a negative impact on innovation, which would be achieved by a 'partially connected' network structure.
Indeed, human societies typically develop partially connected networks that allow them to acquire relatively large effective population sizes while maintaining local differentiation~\cite{migliano2017-hunterGathererNetwork, padilla2022-interconnectivity}.
Thus, through the coexistence of diverse cooperative-specialist units, humans would have exploited the four properties of evolutionary selection processes.
